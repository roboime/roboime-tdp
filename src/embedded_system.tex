\section{Embedded System}\label{emb_sys_sec}

RoboIME electronics consist of nine boards: (a) the Main Board, responsible for
communication between the other boards; (b) the Stamp Board,responsible for the
embedded computations; (c) the Kicker Board, responsible for maintaining high
voltage and activate the kickers; (d) five Motor Controller Boards which are
responsible for the robot's motion control and the dribbling device; (e) the
Transceiver Board, which is responsible for the link between the robot and the
main computer. All these boards are describes below in this section.


\subsection{Main Board}

The Main Board features a socket to plug the boards in: the kicker's sensor, a
optical sensor used to detect if the robot has the ball possession; dribbler
motor, which makes possible to the robot to spin and move backward without
losing the ball; four quadrature encoders and the power supply with safety
devices.


\subsection{Stamp Board}

This board is responsible for performing all the logical function. Serving as a
brain for the electronic system there is an embedded STM32F407VG
microcontroller, with an ARM Cortex M4 as main CPU, 1 MB Flash, 192 KB RAM
memory working at 168 MHz, that was programmed in C using CoIDE and Eclipse
IDEs. The main function of the embedded system is to receive data from the AI
and convert into movement. For do so, there is a Proportional Derivative
Integrative Control sampling the real wheel's velocity, comparing with the
desired and outputting the appropriate voltage to the motor.
That control has fundamental importance in looking for the correct velocity of
the wheel. There is also a current control that avoid the Motor Controller
Board to burn out.

%vide http://www.st.com/st-web-ui/static/active/en/resource/sales_and_marketing/promotional_material/brochure/brstm32f4.pdf


\subsection{Kicker Board}

This board is responsible for produce the high voltage used to activate the two
coils, controlling the kick strength and discharge almost instantly all the
power stored on the coils. There are two kinds of kick, the forward kick and
the high kick. There are two steps in this board: charge and discharge. The
first one has the unique function of keep a constant output of 180V DC from an
input of 7/8V DC. A DC-DC step-up power supply controlled by the MC34063 IC and
two electrolytic capacitors of 2200F, 200V are used for this task. The second
one is to drive the kickers. In this part are used one TC4427 MOSFET Driver IC
and two IRFP4868PBF Power MOSFETs that are responsible for close the high
voltage circuit of the first step with the ground through the coil, converting
electrical into mechanical energy. A precise control of the actuation time
ensures that the kick will occur with the right velocity.

%include image of high kick


\subsection{Motor Controller Board}

The idea of the RoboIME electronic is to modularize the electronic project. For
this, there is one controller module board for each wheel motor. If one of them
burns out, it is possible to exchange it quickly. Each board has two TC4427
(MOSFET driver) and two IRF7319 (complementary half H bridge). These ICs create
an H-bridge, allowing the velocity control in both directions through a
Pulse-Width Modulation, converting a digital signal input into an analog
output.


\subsection{Transceiver Board}

The Transceiver Board is responsible for the link between the microcontroller
and the main computer through a nRF24L01 wireless chip. The first is done using a
own protocol, operating in the 2.4GHz band, simplex, fully compliant with FCC and
ETSI regulations. The second is accomplished through the Serial Peripheral
Interface Bus (SPI), a standard.

% vim: tw=79 et sw=4 ts=4 filetype=tex
